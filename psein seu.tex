\documentclass[letterpaper,12pt]{article}
\usepackage{geometry}
\geometry{left=2 cm,right=2cm, top=3cm, bottom=2cm}
\usepackage{currvita} 
\usepackage[utf8]{inputenc}
\usepackage[spanish]{babel}
\usepackage{graphicx}
\usepackage{amssymb}
\title{PSEINT ALGORITMO  POTENCIACION }
\date{28 de julio 2020}
\setlength{\parskip}{0.1cm}
\pagestyle{empty}

\begin{document}
\maketitle

 \begin{cvlist}{Autor}
	\item[Nombre completo]  JOSE MANUEL DIAZ POSADA
	\item[Código]201960133 
	\item[Programa Academico]TECNOLOGIA EN ELCTRONICA 
	\item[Correo electrónico] jose.manuel.diaz@correounivalle.edu.co
\end{cvlist}

Para este informe realice un algoritmo en pseint, se mostrara toda su estrutura que lo compone y se realizar la prueba de escritorio y  mostrare Un diagrama de flujo ya que se puede apreciar de manera más visual el procesamiento que realiza la compilacion y el paso a paso de ejecución, este utiliza  figuras geometricas para hacerse más interactivo
\begin{verbatim}

Algoritmo de potencia

     SubProceso  resultado <- Potencia (base, exponente)
               Si exponente=0 Entonces
             resultado <- 1;
                sino 
                resultado <- base*Potencia(base,exponente-1); 
                     FinSi
       FinSubProceso

           Proceso sabernumeroelevado 
     Escribir "Ingrese Base"
         Leer base
    Escribir "Ingrese Exponente"
    Leer exponente
         Escribir "El resultado es ",Potencia(base,exponente)
   FinProceso

se pudo observar la estrutura del codigo .
\end{verbatim}
el siguiente paso  le daremos correr al codigo para ver si funciona adecuadamente.
como se muestra en la figura 1 es correcto el algoritmo.
\begin{figure}[h!]
\centering
\includegraphics[width=17cm, height=7.5cm]{jos.jpg}
\caption{proceso compilacion}
\label{Figura: 1}
\end{figure}

\begin{figure}[h!] 
\centering
\includegraphics[width=17cm, height=7.5cm]{diagra.jpg} 
\caption{Diagrama de flujo}
\label{Figura: 2}
\end{figure}

\textrm {se puede observar el diagrama deflujo es un metodo universal para entender un codigo no importa el idioma con el diagrama lo podemos entender de una manera mas agradable y nos ayuda a ver como esta echo el codigo .}

 podemos concluir con la prueba de escritorio  y deducir que nuestro algoritmo
funciona correctamente y cumple adecuadamente con el diagrama de frujo.

\begin{figure}[h!] 
\centering
\includegraphics[width=17cm, height=9.5cm]{prueba.jpg} 
\caption{prueba de escritorio}
\label{Figura: 3}
\end{figure}






\end{document}